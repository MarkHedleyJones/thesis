%!TEX root = ./thesis.tex

\documentclass[11pt,a4paper,oneside,english]{book}
\usepackage{mathpazo}
\usepackage{helvet}
\usepackage{courier}
\renewcommand{\familydefault}{\rmdefault}
\usepackage[T1]{fontenc}
\usepackage[utf8]{inputenc}
\pagestyle{headings}
\setcounter{secnumdepth}{3}
\setcounter{tocdepth}{3}
\usepackage[english]{babel}
\usepackage{verbatim}
% \usepackage{prettyref}
% \usepackage{refstyle}
% \usepackage{fixltx2e}

\usepackage[version=3]{mhchem}
\usepackage{float}
\usepackage{rotfloat}
\usepackage{textcomp}
\usepackage{url}
\usepackage[font=small,labelfont=bf]{caption}
\usepackage{amsmath}
\usepackage{amssymb}
\usepackage{graphicx}
\usepackage{setspace}
\usepackage{siunitx}
\usepackage{multirow}
% \usepackage{nomencl}
\usepackage{cite}


% the following is useful when we have the old nomencl.sty package
\providecommand{\printnomenclature}{\printglossary}
\providecommand{\makenomenclature}{\makeglossary}
\makenomenclature
\doublespacing
\usepackage[unicode=true,pdfusetitle,
 bookmarks=true,bookmarksnumbered=false,bookmarksopen=false,
 breaklinks=false,pdfborder={0 0 1},backref=section,colorlinks=false]
 {hyperref}

\makeatletter

%%%%%%%%%%%%%%%%%%%%%%%%%%%%%% LyX specific LaTeX commands.

\AtBeginDocument{\providecommand\partref[1]{\ref{part:#1}}}
\pdfpageheight\paperheight
\pdfpagewidth\paperwidth

%% Because html converters don't know tabularnewline
\providecommand{\tabularnewline}{\\}
\floatstyle{ruled}
\newfloat{algorithm}{tbp}{loa}[chapter]
\providecommand{\algorithmname}{Algorithm}
\floatname{algorithm}{\protect\algorithmname}
% \RS@ifundefined{subref}
%   {\def\RSsubtxt{section~}\newref{sub}{name = \RSsubtxt}}
%   {}
% \RS@ifundefined{thmref}
%   {\def\RSthmtxt{theorem~}\newref{thm}{name = \RSthmtxt}}
%   {}
% \RS@ifundefined{lemref}
%   {\def\RSlemtxt{lemma~}\newref{lem}{name = \RSlemtxt}}
%   {}


\@ifundefined{date}{}{\date{}}
%%%%%%%%%%%%%%%%%%%%%%%%%%%%%% User specified LaTeX commands.
\usepackage[toc,page]{appendix}
\usepackage{listings}%for inserting source code
\usepackage{microtype}% makes pdf look better
\usepackage{sectsty}% Changing section headinds
\usepackage{booktabs}
\usepackage{longtable}
\usepackage{color}
\usepackage{anyfontsize}
\definecolor{chaptergrey}{rgb}{0.8,0.8,0.8}
\definecolor{dark}{rgb}{0.2,0.2,0.2}
\usepackage[helvetica,nogrey]{quotchap}
\linespread{1.05}        % Palatino needs more leading
%\usepackage[euler-digits]{eulervm}
%\renewcommand{\rmdefault}{pplx}

\usepackage{titlesec}
\titleformat{\chapter}[block]
  {\Huge\sffamily\huge\bfseries\color{chaptergrey}}
  {\sffamily\fontsize{40}{190}\selectfont Chapter \thechapter}{0pt}{\\\color{dark}}

\titleformat{\part}[block]
  {\Huge\sffamily\huge\bfseries\color{chaptergrey}}
  {\sffamily\fontsize{100}{0}\selectfont Part \thepart}{0pt}{\\\centering\color{dark}}


\titleclass{\part}{top}

\lstset{ %
basicstyle=\footnotesize,       % the size of the fonts that are used for the code
numbers=left,                   % where to put the line-numbers
numberstyle=\footnotesize,      % the size of the fonts that are used for the line-numbers
numbersep=5pt,                  % how far the line-numbers are from the code
backgroundcolor=\color{white},  % choose the background color. You must add \usepackage{color}
showspaces=false,               % show spaces adding particular underscores
showtabs=false,                 % show tabs within strings adding particular underscores
frame=single,                   % adds a frame around the code
tabsize=2,                      % sets default tabsize to 2 spaces
captionpos=b,                   % sets the caption-position to bottom
breaklines=true,                % sets automatic line breaking
breakatwhitespace=false,        % sets if automatic breaks should only happen at whitespace
%title=\lstname,                % show the filename of files included with \lstinputlisting;
                                % also try caption instead of title
escapeinside={\%*}{*)},         % if you want to add a comment within your code
morekeywords={*,...}            % if you want to add more keywords to the set
}

% Fuzz
\hfuzz 2pt % Don't bother to report over-full boxes if over-edge is < 2pt

\hypersetup{
    colorlinks,
    citecolor=black,
    filecolor=black,
    linkcolor=black,
    urlcolor=blue
}

\makeatother

\usepackage{listings}
\addto\captionsbritish{\renewcommand{\algorithmname}{Algorithm}}
\addto\captionsbritish{\renewcommand{\lstlistingname}{Listing}}
\renewcommand{\lstlistingname}{Listing}

\usepackage{cleveref}
