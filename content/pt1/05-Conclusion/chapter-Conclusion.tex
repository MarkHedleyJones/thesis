%!TEX root = ../../../thesis.tex

Estimates of energy availability have been made by calculating the amount of energy lost in a traditional water meter.
Those estimates showed that for a typical New Zealand household approximately \SI{280}{\joule} is available for capture every day.
In the previous section, the amount of energy that would be needed to run an electronic water meter was estimated to be \SI{12}{\joule} per day.
This means that in order to power an electronic water meter a minimum conversion efficiency of \SI{4.28}{\percent} is required.
%However, calculations from cells assembled in \cref{sect:part1_energyHarvesting_measuringStreamingCells} showed that readily obtainable conversion efficiencies are in the order of \SI{0.28}{\micro\percent}.
However, calculations from cells assembled in \cref{sect:part1_energyHarvesting_measuringStreamingCells} showed that readily obtainable conversion efficiencies are in the order of \SI{2.8e-9}.
Conversion efficiencies over \SI{1}{\percent} have not been reported in the literature, and one paper suggests the theoretical maximum is \SI{2}{\percent} \cite{VanderHeyden2006}.
For these reasons, the use of streaming cells as a method of energy harvesting from water and current materials is expected to be infeasible.
The literature suggests there is room for improvement, to levels which would make streaming cell harvesters practical, but these gains are reliant on new nano-materials.

During the course of this research two issues came to light with regards to streaming cell harvesting.
The first issue is a susceptibility to clogging.
Having such narrow openings in a domestic water feed is likely to trap dirt and contaminants at the channel openings.
This lowers the effective efficiency and will require periodic cleaning, lowering the benefit to utility companies.
The second issue is the manufacturing precision required to create the channels.
Parts manufactured with high precision are generally small, but a streaming cell would need precise dimensions and a large surface area.
This could be overcome with the use of materials like porous glass.
As a result of the low measured efficiency, streaming cells for the purpose of energy harvesting are not studied further.
\Cref{part:doubleLayersOnConductors} looks at the electrical impedance of medical implant electrodes.
The research here into double layers and the role they play in energy harvesting applications is directly applicable there.

% Edit checkpoint 2015-09-17 19:38