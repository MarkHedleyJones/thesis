%!TEX root = ../../thesis.tex


%% INTRODUCTION


Is it possible to harvest energy from water without moving parts?
What is the electrical impedance between electrodes in an electrolyte solution?
Although seemingly unrelated, the answer to both lies in behaviour that occurs where liquids comes into contact with solids.
That behaviour is the formation of arranged layers of liquid against the solid surface, called a double layer.
This thesis is separated into two parts, each addressing one of the two questions above related to double layers.
\Cref{part:doubleLayersOnInsulators} studies double layers on insulating solids as a means of energy conversion.
A number of double layer based power harvesters are fabricated and their output is measured.
Converting fluid energy into electrical using double layers would allow for a ``no moving parts'' or ``solid-state'' energy harvester.
Such a harvester could potentially outlast a mechanically based equivalents (due to reduced wear on moving components) and be cheaper to produce (owing to a lower component count).
One application of particular interest is smart metering of domestic water usage.
\Cref{part:doubleLayersOnConductors} models the electrical impedance between two electrodes when submerged in an electrolyte.
Double layers play a large role in this impedance as they dictate the concentration of ions at the electrode's surface.
Measurement of interface impedance allows for direct comparison between a range of environments into which electrodes are placed.
This is important when designing an implant that will be inserted into a person.
Before introducing background material on interfacial double layers, my motivation for doing this work is discussed.
This is followed by a statement of originality and an outline of the structure of this thesis.


\section{Motivation}
  \label{sect:introduction_motiviation}


  My research began with the question ``is it possible to harvest energy from water without moving parts?''
  The motivation to answer this question lay in the idea of building an energy harvester to power an electronic water meter.
  Doing this without the moving parts of more traditional mechanisms, such as turbines, should increase the harvester's life-span and be generally more robust.
  I started by looking at three possible harvesting mechanisms:
  \begin{itemize}
    \item piezoelectric oscillators,
    \item electrostatic generators, and
    \item streaming potential cells
  \end{itemize}
  The piezoelectric oscillator was the equivalent of a water whistle with a vibrational energy harvester attached.
  The electrostatic generator was a version of Lord Kelvin's Electrostatic Generator with a harvesting application~\cite{Thomson1867a}.
  And the streaming potential cell was a mystery at the time.
  We knew geologists used streaming potentials to measure underground water flow.
  The only thing we knew about the mechanism was that forcing water through something somehow generated a voltage.
  Learning about that mechanism and answering the following questions started me on the path that became this thesis.
  \begin{enumerate}
    \item Where does streaming voltage come from?
    \item What role does the geometry of a streaming device play?
    \item Could you change the materials to get more voltage?
  \end{enumerate}
  After experimentation and energy budgeting, I eventually concluded that streaming cell harvesters are not yet practical.
  Low conversion efficiency, a susceptibility to clogging and the need for high manufacturing tolerances make them unsuited for domestic water metering.
  However, this research allowed me to gain  a working knowledge of interfacial double layers.

  During my doctoral studies my supervisor, Jonathan Scott, took a sabbatical at Saluda Medical in Sydney.
  At the time, Saluda were developing a medical implant for spinal cord stimulation.
  Jonathan and Saluda's senior electronic engineer developed an electrical model of the impedance presented by electrodes immersed in a solution of saline.
  That model uses electrical components to simulate the electrical impedance between an electrode and an electrolyte.
  This means it can be entered into electrical simulation software and used to simulate an implanted electrode.
  Much of the behaviour the model simulates is due to double layers.
  Saluda's engineers use a dilute solution of phosphate buffered saline to approximate human spinal cavities into which their electrodes are implanted.
  They do not know how good this approximation is, but it was the most appropriate mixture they had.
  The alternative was to embed an electrode in a live animal and measure the response - that is also what they do.
  Live animal experiments are costly and how they differ from solutions of saline is still unknown.
  The interface model is the starting point for the second phase of my research, which characterises the interface between an electrode and biological solutions.
  I have leveraged my understanding of interfacial double layers from \cref{part:doubleLayersOnInsulators} to understand how the model worked, and use it correctly.


\section{Statement of Originality}

  The work contained in this thesis is my own except where otherwise acknowledged.
  % Measurements of the energy consumed during an EEPROM write, an ADC measurement, a single instruction being executed, and during sleep mode for six 8-bit microprocessors are my own.
  % The relationship between an electrolyte's conductivity and the impedance of the constant phase element, presented in~\Cref{part:doubleLayersOnConductors}, is my own.
  % The recipe for a mixture that improves the match between live sheep spine and saline is my own.
  % The measurement configuration for sampling Faradaic current which removes the effect of double layer capacitance between electrodes in an electrolyte is my own.


\section{Publications Arising From This Work}


  \begin{itemize}
    \item Jones, M.H. \& Scott, J. (2014). \emph{Scaling of Electrode-Electrolyte Interface Model Parameters In Phosphate Buffered Saline.} Published in IEEE Transactions on Biomedical Circuits and Systems, Issue 99.
    \item Jones, M.H. \& Scott, J. (2014). \emph{Feasibility of Harvesting Power to Run a Domestic Water Meter Using Streaming Cell Technology.} In proceedings of the 21st Electronics New Zealand Conference, ENZCON 2014, Waikato University, Hamilton, New Zealand.
    \item Jones, M.H. \& Scott, J.B. (2011). \emph{The energy efficiency of 8-bit low-power microcontrollers.} In Proceedings of the 18th Electronics New Zealand Conference, ENZCON 2011, Massey University, Palmerston North, 21-22 November 2011, pp. 87-90.
  \end{itemize}


\section{Thesis Outline}


  This thesis is broken into two parts.
  \Cref{part:doubleLayersOnInsulators} is concerned with energy harvesting with double layers, specifically by the use of streaming cells.
  \Cref{part:doubleLayersOnConductors} measures and models the impedance of an interface, specifically those between implant electrodes.
  Put simply, \cref{part:doubleLayersOnInsulators} deals with double layers on insulating surfaces, and \cref{part:doubleLayersOnConductors} deals with double layers on conductive surfaces.

  The next chapter (\cref{chap:background}) contains background material on double layers, including their formation and a breakdown of their structure.
  The topics of streaming cells and impedance modelling are introduced in that chapter.
  Then, \cref{part:doubleLayersOnInsulators} begins by looking at streaming cells for the purpose of running an energy harvesting water meter.
  It starts at \cref{chap:part1_streamingCellHarvesters} with a brief mathematical analysis, followed by streaming cell fabrication, and then measurements of their ability to harvest energy.
  \Cref{chap:part1_waterMetering} estimates the amount of energy that would be available to a streaming cell energy harvester.
  \Cref{chap:part1_energyHarvestingRequirements} looks at the amount of energy needed to run microprocessors and wireless transmitters.
  This concludes with an estimate of the amount of energy required to run an electronic water meter.
  To conclude \cref{part:doubleLayersOnInsulators}, \cref{chap:part1_conclusion} combines the data obtained and the feasibility of streaming cell energy harvesting for electronic water metering is discussed.

  The second part of the thesis (\cref{part:doubleLayersOnConductors}) starts with an overview of the electrode interface model (\cref{chap:theInterfaceModel}).
  \Cref{chap:interfaceParameters} deals with measurement of the various model parameters in both phosphate buffered saline (\cref{sect:pbs_measurements}) and inside a live sheep's spinal cavity (\cref{sect:sheep_measurements}).
  Finally, \cref{chap:fluid_mimicry} presents work on the creation of a mixture designed to better represent the environment inside sheep spine compared to phosphate buffered saline (PBS).
